% Für Bindekorrektur als optionales Argument "BCORfaktormitmaßeinheit", dann
% sieht auch Option "twoside" vernünftig aus
% Näheres zu "scrartcl" bzw. "scrreprt" und "scrbook" siehe KOMA-Skript Doku
\documentclass[12pt,a4paper,titlepage,headinclude,bibtotoc]{scrartcl}


%---- Allgemeine Layout Einstellungen ------------------------------------------

% Für Kopf und Fußzeilen, siehe auch KOMA-Skript Doku
\usepackage[komastyle]{scrpage2}
\pagestyle{scrheadings}
\setheadsepline{0.5pt}[\color{black}]
\automark[section]{chapter}


%Einstellungen für Figuren- und Tabellenbeschriftungen
\setkomafont{captionlabel}{\sffamily\bfseries}
\setcapindent{0em}


%---- Weitere Pakete -----------------------------------------------------------
% Die Pakete sind alle in der TeX Live Distribution enthalten. Wichtige Adressen
% www.ctan.org, www.dante.de

% Sprachunterstützung
\usepackage[ngerman]{babel}

% Benutzung von Umlauten direkt im Text
% entweder "latin1" oder "utf8"
\usepackage[utf8]{inputenc}

% Pakete mit Mathesymbolen und zur Beseitigung von Schwächen der Mathe-Umgebung
\usepackage{latexsym,exscale,stmaryrd,amssymb,amsmath}

% Weitere Symbole
\usepackage[nointegrals]{wasysym}
\usepackage{eurosym}

% Anderes Literaturverzeichnisformat
%\usepackage[square,sort&compress]{natbib}

% Für Farbe
\usepackage{color}

% Zur Graphikausgabe
%Beipiel: \includegraphics[width=\textwidth]{grafik.png}
\usepackage{graphicx}

% Text umfließt Graphiken und Tabellen
% Beispiel:
% \begin{wrapfigure}[Zeilenanzahl]{"l" oder "r"}{breite}
%   \centering
%   \includegraphics[width=...]{grafik}
%   \caption{Beschriftung} 
%   \label{fig:grafik}
% \end{wrapfigure}
\usepackage{wrapfig}

% Mehrere Abbildungen nebeneinander
% Beispiel:
% \begin{figure}[htb]
%   \centering
%   \subfigure[Beschriftung 1\label{fig:label1}]
%   {\includegraphics[width=0.49\textwidth]{grafik1}}
%   \hfill
%   \subfigure[Beschriftung 2\label{fig:label2}]
%   {\includegraphics[width=0.49\textwidth]{grafik2}}
%   \caption{Beschriftung allgemein}
%   \label{fig:label-gesamt}
% \end{figure}
\usepackage{subfigure}

% Caption neben Abbildung
% Beispiel:
% \sidecaptionvpos{figure}{"c" oder "t" oder "b"}
% \begin{SCfigure}[rel. Breite (normalerweise = 1)][hbt]
%   \centering
%   \includegraphics[width=0.5\textwidth]{grafik.png}
%   \caption{Beschreibung}
%   \label{fig:}
% \end{SCfigure}
\usepackage{sidecap}

% Befehl für "Entspricht"-Zeichen
\newcommand{\corresponds}{\ensuremath{\mathrel{\widehat{=}}}}
% Befehl für Errorfunction
\newcommand{\erf}[1]{\text{ erf}\ensuremath{\left( #1 \right)}}

%Fußnoten zwingend auf diese Seite setzen
\interfootnotelinepenalty=1000

%Für chemische Formeln (von www.dante.de)
%% Anpassung an LaTeX(2e) von Bernd Raichle
\makeatletter
\DeclareRobustCommand{\chemical}[1]{%
  {\(\m@th
   \edef\resetfontdimens{\noexpand\)%
       \fontdimen16\textfont2=\the\fontdimen16\textfont2
       \fontdimen17\textfont2=\the\fontdimen17\textfont2\relax}%
   \fontdimen16\textfont2=2.7pt \fontdimen17\textfont2=2.7pt
   \mathrm{#1}%
   \resetfontdimens}}
\makeatother

%Honecker-Kasten mit $$\shadowbox{$xxxx$}$$
\usepackage{fancybox}

%SI-Package
\usepackage{siunitx}

%keine Einrückung, wenn Latex doppelte Leerzeile
\parindent0pt

%Bibliography \bibliography{literatur} und \cite{gerthsen}
%\usepackage{cite}
\usepackage{babelbib}
\selectbiblanguage{ngerman}

\begin{document}

\begin{titlepage}
\centering
\textsc{\Large Anfängerpraktikum der Fakultät für
  Physik,\\[1.5ex] Universität Göttingen}

\vspace*{3cm}

\rule{\textwidth}{1pt}\\[0.5cm]
{\huge \bfseries
  Versuch Nr. 24 Radioaktivität\\[1.5ex]
  Protokoll}\\[0.5cm]
\rule{\textwidth}{1pt}

\vspace*{3cm}

\begin{Large}
\begin{tabular}{ll}
Praktikant: &  Michael Lohmann\\
 &  Felix Kurtz\\
% &  Kevin Lüdemann\\
% &  Skrollan Detzler\\
 E-Mail: & m.lohmann@stud.uni-goettingen.de\\
 &  felix.kurtz@stud.uni-goettingen.de\\
% &  kevin.luedemann@stud.uni-goettingen.de\\
 Betreuer: & Phillip Bastian\\
 Versuchsdatum: & 12.03.2015\\
\end{tabular}
\end{Large}

\vspace*{0.8cm}

\begin{Large}
\fbox{
  \begin{minipage}[t][2.5cm][t]{6cm} 
    Testat:
  \end{minipage}
}
\end{Large}

\end{titlepage}

\tableofcontents

\newpage

\section{Einleitung}
\label{sec:einleitung}
Radioaktivität ist eine der wichtigsten Voraussetzungen für die Vielfalt der Oberfläche unserer Erde.
Ohne sie wäre der Erdkern kalt und es gäbe viele Gesteine wie Marmor nicht, da kein Vulkanismus existieren könnte.
Außerdem stellt sie eine Gefahr für den Menschen dar, da sie das Erbgut verändern und so Zellen schädigen kann.
Daher ist es von existentieller Bedeutung, sie zu verstehen.
Dies soll mit diesem Versuch geschehen.
\cite{lp24}
 
\section{Theorie}
\label{sec:theorie}
\subsection{Radioaktiver Zerfall}
Sind die Atomradien zu groß, so zerfallen die Atomkerne, da die Starke Kernkraft nicht mehr ausreicht, die Protonen zusammenzubinden.
Der Zerfall kann auf drei Weisen stattfinden:
\begin{itemize}
\item $\alpha$-Zerfall, bei dem Heliumkerne (\chemical{^4_2He^{2+}}) emittiert werden
\item $\beta$-Zerfall, bei dem entweder Elektronen ($e^-$ bei dem $\beta^-$) oder Positronen ($e^+$ bei dem $\beta^+$) ausgesandt werden
\item $\gamma$-Zerfall, bei dem energiereiche Photonen (Röntgenstrahlung) ausgesand wird
\end{itemize}

\subsection{Silberisotope}
Die beiden Isotope \chemical{^{107}Ag} und \chemical{^{109}Ag} sind stabil.
Sie können jedoch durch Anregung mit einem Neutron in die Isotope \chemical{^{108}Ag} und \chemical{^{110}Ag} umgewand werden.
Diese sind jedoch $\beta^-$-Strahler, welche beim Zerfall $\gamma$-Strahlung aussenden.
Diese kann in einem \textsc{Geiger-Müller}-Zählrohr gemessen werden.

                                                                                                                                                                      
\section{Durchführung}
\label{sec:durchfuehrung}
Zunächst wird der Computer zur Auswertung hochgefahren und das Programm zur Datenspeicherung geöffnet.
Das Ausgabegerät des Geigerzählers wird angeschaltet.
Nun kann das Silberplättchen mit einer Pinzette in den langen Halter gelegt werden, in dem es in die radioaktive Quelle geführt wird.
Zeitgleich wird eine Stopuhr gestartet, die die Zeit misst, wie lange es aktiviert wird.
Diese beträgt 1,2,4 und $8\si\minute$.
Danach wird der lange Halter herausgezogen und das Plättchen erneut mit der Pinzette in den kurzen Halter gelegt.
Während des Herausziehens wird am Computer der Knopf "`Zeit starten"' betätigt.
Anschließend wird die Probe zügig zu dem Geiger-Müller-Zählrohr getragen und sobalt sie eingeführt wurde wird am Computer die Messung gestartet.
Nun muss gewartet werden, bis die Zählrate pro Sekunde ungefähr konstant ist.
Dies tritt bei einem Wert von ungefähr 5 Zerfällen pro Sekunde ein.

\section{Auswertung}
\label{sec:auswertung}

\section{Diskussion}
\label{sec:diskussion}

\bibliography{literatur}
\bibliographystyle{babalpha}
\end{document}
